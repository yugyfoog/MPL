\documentclass{article}
\begin{document}
\title{The MPL Programming Language}
\author{Mark Kamradt}
\maketitle
\section{Introduction}
\emph{MPL}\footnote{MPL is an ambigius initialism.
  For most people it could stand for Mathematical Programming Language.
  If you're someone who know's me it stands for Mark's Programming Language.
  To me it's just My Programming Language.}
is a simple impartative mathematical scripting language.
The intent of this language is for writting small mathematical programs.
Programs that are to small to need the facilities of a real programming
language, yet to large for a handheld calculator.

If you type a mathematical expression, \emph{MPL} will respond with
the result.  For example,
\begin{verbatim}
(1 + sqrt(5))/2
\end{verbatim}
gives
\begin{verbatim}
1.61803
\end{verbatim}

You can also save the results of an expression into a variable.
\begin{verbatim}
phi = 0.5 - 0.5*5^0.5
\end{verbatim}
And then use that variable in expressions
\begin{verbatim}
-1/phi
\end{verbatim}

You can also define functions. For example to compute the $n$th Fibonacci
number define the function
\begin{verbatim}
function fibonacci(n)
    q = (1 + sqrt(5))/2
    p = (1 - sqrt(5))/2
    return (q^n - p^n)/sqrt(5)
end
\end{verbatim}
and then use that function in an expression
\begin{verbatim}
fibonacci(101)/fibonacci(100)
\end{verbatim}
\section{Lexical}
\emph{MPL} is a line oriented language.
Each statement is on its own line.
Ending a line with a backslash, \verb|\|, will concatenate the next line
to the end of the current line.
Comments start with a tilde, \verb|~|, and continue to the end of the line.

Each line is divided into \emph{word symbols}, \emph{numbers},
\emph{strings}, and \emph{character symbols}.
Word symbols start with an alphabetic character and are followed by
zero or more alphanumeric characters.
The underscore, \verb|_|, and the dollar sign, \verb|$|, count as
alphabetic characters.

There are no reserved words in \emph{MPL} but words with special meaning
to the language should be avoided for naming variables and functions.
Also words that begin with an underscore should be avoided.

Numbers can be written as unsigned integers, decimal numbers or
in scientific notation.

Strings literals start and end with a double quote, \verb|"|,
and must be on a single line. The C language escape characters
can be used except octal and hex codes.

\section{Types}
There are eight data types in MPL. \emph{Real}, \emph{complex},
\emph{real vector}, \emph{complex vector}, \emph{real matrix}, \emph{complex matrix}, \emph{string}, and \emph{list}. There are no facilities for creating
addition data types.

\subsection{Real}
The \emph{real} data type is a double-precision floating point value.\footnote{Hopefully IEEE 754.}

The math operators for \emph{real} are \verb|+| (addition),
\verb|-| (subtraction and negation), \verb|*| (multiplication),
\verb|/| (division), \verb|div| (floored division), \verb|mod| (modulus),
and \verb|^| (exponenazation).
Floored division is $q = \lfloor a/b \rfloor$.
The modulus operation is then $r = b\lfloor a/b \rfloor - a$.
Note that these are different than most other languages.

\emph{Real} numbers can be compared using the relational operators
equal \verb|==|, not equal \verb|!=|, less than \verb|<|, less than or equal \verb|<=|, greater than \verb|>|, and greater then or equal \verb|>=|.  The results of the relational operators is either 1.0 for true and 0.0 for false.

Since \emph{MPL} has no Boolean type the Boolean operators use type \emph{real}.
Any non-zero value is treated as true and zero is reated as false.
The results are alway either 1.0 for true and 0.0 for false.
The Boolean operators are \verb|and|, \verb|or|, and \verb|not|.
The constants \verb|TRUE| and \verb|FALSE| are defined to 1 and 0.

\subsection{Complex}

The \emph{complex} type is represented by a pair of double precision floating point numbers. They can by typed by surrounding the comma sepreated real and imaginary parts with parenthesis. The number $3 + 4i$ is writen
\begin{verbatim}
(3,4)
\end{verbatim}
The values for the parts of a complex number can be any \emph{real} valued
expression.
You can used the notation for constructing complex number out of
real number parts.
\begin{verbatim}
(cos(x)^2, sqrt(y))
\end{verbatim}
The constant \verb|I| is defined as the imanginary unit, $i$. You can use this for a more traditional
representation of a complex number.
\begin{verbatim}
3 + 4*I
\end{verbatim}

To specify a complex number in polar notation there are a few different
ways.
\begin{verbatim}
polar(2, PI/6)
2*cis(PI/6)
2*exp(I*PI/6)
\end{verbatim}
all represent the same number.

To get the real or imaginary parts of a complex number use
the \verb|real| or \verb|imag| functions.
To get the magnitude or argument of a complex number use the
\verb|abs| or the \verb|arg| functions.
The \verb|norm| function returns the magnitude squared of a
complex number. To find the complex conjugate of a number
use the \verb|conj| function.

\subsection{Vector}

The vector type is a Euclidean $n$-space vector.
Vectors are not ment to be used as general purpose arrays.

The math operators for vector are \verb|+| (addition),
\verb|-| (subtraction and negation), \verb|*| (scalar multiplication,
dot product), \verb|/| (scalar division). 


\section{Expressions}
\section{Commands}
\subsection{Print Command}
The \emph{print} command is just an expression on a line by itself.
\begin{center}
  \verb|<|\emph{expression}\verb|>|
\end{center}
The expression is evaluated and its value printed.
For formated ouput see the standard library.
\subsection{Assignment Command}
The \emph{assignemnt} command evaluates an expresion and
assigns its value to a variable.
\begin{center}
  \verb|<|\emph{lvalue}\verb|> = <|\emph{expression}\verb|>|
\end{center}
The \emph{lvalue} can be either a simple variable or an indexed variable.
If the \emph{lvalue} is a simple variable the previous value and
type of the variable are discarded for the new value.
If the \emph{lvalue} is an indexed variable the type (including sizes)
of the expression must match the lvalue.
\subsection{Include Command}
The \emph{include} command is use to the include a file as if
that file replaces the include command.
\begin{center}
  \verb|include <|\emph{expression}\verb|>|
\end{center}
\section{Statements}
\section{Functions}
\section{Standard Library}
\section{Things You're Not Suppose to Know.}
This section is for documenting ``undocumneted'' features of \emph{MPL}.
These are features of the language that may change in the future.
\subsection{Real Numbers}
\emph{Real} numbers can be written in hexadecimal form.
A \verb|#| (pound symbol) followed by hexidecimal digits give the bits
of the floating point number. For example, \verb|#3ff0000000000000|
represents the number 1.

The way the current interpreter reads \emph{real} literals,
a \verb|.| (decimal point) by itself represents the number
zero.

\end{document}
