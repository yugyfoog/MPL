\documentclass{article}
\begin{document}
\title{The MPL Programming Language}
\author{Mark Kamradt}
\maketitle
\section{Introduction}
\emph{MPL}\footnote{MPL is an ambigius initialism.
  For most people it could stand for Mathematical Programming Language.
  If you're someone who know's me it stands for Mark's Programming Language.
  To me it's just My Programming Language.}
is a simple impartative mathematical scripting language.
The intent of this language is for writting small mathematical programs.
Programs that are to small to need the facilities of a real programming
language, yet too large for a handheld calculator.

\emph{MPL} is meant to be a scripting lauguage.
The ideal workflow is: find a problem, collect data for input,
write a script in you favorite text editor, run your script,
(maybe debug your script) find your solution.
You could use the interpreter interactivaly, but it doesn't have
any of the niceties of a user friendly interpreter.

If you type a mathematical expression, \emph{MPL} will respond with
the result.  For example,
\begin{verbatim}
(1 + sqrt(5))/2
\end{verbatim}
gives
\begin{verbatim}
1.61803
\end{verbatim}

You can also save the results of an expression into a variable.
\begin{verbatim}
phi = (1 - sqrt(5))/2
\end{verbatim}
And then use that variable in expressions
\begin{verbatim}
-1/phi
\end{verbatim}

You can also define functions. For example, to compute the $n$th Fibonacci
number, define the function:
\begin{verbatim}
function fibonacci(n)
    q = (1 + sqrt(5))/2
    p = (1 - sqrt(5))/2
    return (q^n - p^n)/sqrt(5)
end
\end{verbatim}
and then use that function in an expression
\begin{verbatim}
fibonacci(101)/fibonacci(100)
\end{verbatim}

There are several data type built-in to \emph{MPL}.
In addtion to real (the same as C's double) there are also
complex, real and complex vectors, real and complex matricies,
strings and lists.
The vector and matrix types are \emph{not} general purpose arrays.
The list type can be used as an array.

\section{Running the Interpreter}
A typical \emph{MPL} invocation is
\begin{verbatim}
mpl script <input >output
\end{verbatim}
If no script is given, \verb|mpl| will read code from standard input.
More than one script can be listed on the command line, \verb|mpl| will then
execute each script, in order, as though all the script were a single file.

\section{Lexical}
\emph{MPL} is a line oriented language.
Each statement is on its own line.
If a line is too long you can split it in to two lines by by ending
the first line with a backslash, \verb|\|.
Comments start with a tilde, \verb|~|, and continue to the end of the line.

If a line is too long you can split it into multiple lines by
ending each line, except the last, with a backslash, \verb|\|, character.
Line continuations are considered the same as a space so can only occur
where spaces are allowed.
A string cannot span multiple lines, however you can use line contimuations
and the string concatenation operator, \verb|+|.

Each line is divided into \emph{word symbols}, \emph{numbers},
\emph{strings}, and \emph{character symbols}.
Word symbols start with an alphabetic character and are followed by
zero or more alphanumeric characters.
The underscore, \verb|_|, and the dollar sign, \verb|$|, are treated like as
alphabetic characters.

There are no reserved words in \emph{MPL} but words with special meaning
to the language should be avoided for naming variables and functions.
Also words that begin with an underscore should be avoided.

\section{Types}
There are eight data types in MPL. \emph{Real}, \emph{complex},
\emph{real vector}, \emph{complex vector}, \emph{real matrix}, \emph{complex matrix}, \emph{string}, and \emph{list}. There are no facilities for creating
addition data types.

\subsection{Real}
The \emph{real} data type, internally, is represented by a double-precision floating point value.\footnote{Hopefully IEEE 754.}
Real numbers can be written as a decimal number with an optional exponent.
The following are all valid ways to write real numbers:
\begin{center}
  \begin{tabular}{cc}
    \verb|142857| &  \verb|3.14159| \\
    \verb|6.626e-34| & \verb|6.02214076e23| \\
    \verb|1e+100| \\
  \end{tabular}
\end{center}
You cannot use commas or spaces to format numbers.

The arithmetic operators for \emph{real} are:
\begin{center}
  \begin{tabular}{c}
    \verb|-| \emph{real} \\
    \emph{real} \verb|+| \emph{real} \\
    \emph{real} \verb|-| \emph{real} \\
    \emph{real} \verb|*| \emph{real} \\
    \emph{real} \verb|/| \emph{real} \\
    \emph{real} \verb|div| \emph{real} \\
    \emph{real} \verb|mod| \emph{real} \\
    \emph{real} \verb|^| \emph{real} \\
  \end{tabular}
\end{center}
The \verb|div| operator is a floored division, \verb|a div b| is $\lfloor a/b \rfloor $.
The \verb|mod| operator is the modulus operation, \verb|a mod b| is $b\lfloor a/b\rfloor - a$.
Note that this is different then many other languages.

The comparison operators for reals return a real value of 1 for true and a real value of 0 for false.
\begin{center}
  \begin{tabular}{c}
    \emph{real} \verb|==| \emph{real} \\
    \emph{real} \verb|!=| \emph{real} \\
    \emph{real} \verb|<| \emph{real} \\
    \emph{real} \verb|<=| \emph{real} \\
    \emph{real} \verb|>| \emph{real} \\
    \emph{real} \verb|>=| \emph{real} \\
  \end{tabular}
\end{center}

There are logic operators for real. The operands for the logic operators are non-zero for true and zero for false.
\begin{center}
  \begin{tabular}{c}
    \verb|not| \emph{real} \\
    \emph{real} \verb|and| \emph{real} \\
    \emph{real} \verb|or| \emph{real} \\
  \end{tabular}
\end{center}

\subsection{Complex}
Complex numbers are represented by a pair of double-precision floating-point numbers representing the
real and imaginary parts.
They can be written with parenthesis surrounding the comma sepreated real and imaginary parts.
The number $3 + 4i$ is written
\begin{verbatim}
(3, 4)
\end{verbatim}
The values for the parts of a complex number can be any real valued
expression.
The real and imaginary parts can be any real expression.
\begin{verbatim}
(cos(x)^2, sqrt(y))
\end{verbatim}
The constant \verb|I| is defined as the imanginary unit, $i$. You can use this for a more traditional
representation of a complex number.
\begin{verbatim}
3 + 4*I
\end{verbatim}

To specify a complex number in polar notation there are different options.
\begin{verbatim}
polar(2, PI/6)
2*cis(PI/6)
2*exp(I*PI/6)
\end{verbatim}
all represent the number $2e^{i \pi/6}$.

The arithmetic operators for \emph{complex} are:
\begin{center}
  \begin{tabular}{c}
    \verb|-| \emph{complex} \\
    \emph{complex} \verb|+| \emph{complex} \\
    \emph{complex} \verb|-| \emph{complex} \\
    \emph{complex} \verb|*| \emph{complex} \\
    \emph{complex} \verb|/| \emph{complex} \\
    \emph{complex} \verb|^| \emph{complex} \\
  \end{tabular}
\end{center}
The comparison operators complex numbers are:
\begin{center}
  \begin{tabular}{c}
    \emph{real} \verb|==| \emph{real} \\
    \emph{real} \verb|!=| \emph{real} \\
  \end{tabular}
\end{center}
Operators with mixed real and complex operands act as if the
real operand is complex.
Operators that are not defined for complex types are also
undefined for mix real and complex types.

\subsection{Vector}
The vector type is a Euclidean $n$-space vector.
A vector is written with square brackets surounding a comma seperated
list of real expressions.
\begin{verbatim}
[ 1, 2, 3]
[ sqrt(a), a, a^2 ]
\end{verbatim}
The arithmetic operators for vectors are:
\begin{center}
  \begin{tabular}{c}
    \verb|-| \emph{vector} \\
    \emph{vector} \verb|+| \emph{vector} \\
    \emph{vector} \verb|-| \emph{vector} \\
    \emph{real} \verb|*| \emph{vector} \\
    \emph{vector} \verb|*| \emph{real} \\
    \emph{vector} \verb|*| \emph{vector} \\
    \emph{vector} \verb|/| \emph{real} \\
  \end{tabular}
\end{center}
The \emph{vector} \verb|*| \emph{vertor} operator is a dot product and results in a real.
The comparison operators for vectors are:
\begin{center}
  \begin{tabular}{c}
    \emph{vector} \verb|==| \emph{vector} \\
    \emph{vector} \verb|!=| \emph{vector} \\
  \end{tabular}
\end{center}

\subsection{Complex Vector}
The complex vector is a vector of complex numbers instead of real numbers.
Complex vectors are written the same as real vectors except
at least one of the members must be a complex type.
\begin{verbatim}
[ (1,2), 3 ]
\end{verbatim}

The operators for complex vectors are the same as real vectors.
The dot product for complex vectors produce the Hermitian inner product.
When mixing reals and vectors with complex vectors in expressions the
reals and real vectors are treated like complex and complex vectors.

\subsection{Matrix}
A matrix is a two-dimential array of real numbers.
A matrix is written simular to a vector except different rows of the
matrix are seperated by a vertical bar, \verb$|$.
\begin{verbatim}
[ 1, 2, 3 | 4, 5, 6 | 7, 8, 9]
\end{verbatim}
Matricies can also be written with each row on a seperate line.
\begin{verbatim}
[ 1, 2, 3
  4, 5, 6
  7, 8, 9 ]
\end{verbatim}
There is no way to write a matix with a single row.
If the number of members in each row is different then the number columns is the same as
the row with the most members and missing members are filled with 0.

Like vectors, matrix members can be any real-valued expression.
For example, you could write:
\begin{verbatim}
[ cos(theta), -sin(theta)
  sin(theta), cos(theta) ]
\end{verbatim}

The arithmetic operators for \emph{matrix} are:
\begin{center}
  \begin{tabular}{c}
    \verb|-| \emph{matrix} \\
    \emph{matrix} \verb|+| \emph{matrix} \\
    \emph{matrix} \verb|-| \emph{matrix} \\
    \emph{real} \verb|*| \emph{matrix} \\
    \emph{matrix} \verb|*| \emph{real} \\
    \emph{matrix} \verb|*| \emph{vector} \\
    \emph{matrix} \verb|*| \emph{matrix} \\
    \emph{matrix} \verb|/| \emph{real} \\
  \end{tabular}
\end{center}

The comparison operators for matrix are:
\begin{center}
  \begin{tabular}{c}
    \emph{matrix} \verb|==| \emph{matrix} \\
    \emph{matrix} \verb|!=| \emph{matrix} \\
  \end{tabular}
\end{center}

\subsection{Complex Matrix}
The complex matrix is simular to a matrix but with complex values.
They are bulit the same as a real matrix but at least one member is complex.
The operators for a complex matrix are the same as matrix.
Operators mixing matrix/complex matrix, vector/complex vector and real/complex
will treat the real types as their complex counterpart.
Although complex vectors use a Hermitian inner product, complex matrices do not.

\subsection{String}
A string is an array of zero or more characters.
String are written by surounding characters with double quotes.
\begin{verbatim}
"Hello, world"
\end{verbatim}
The quote character and some control characters can be included
in a string with ``C''-style escape sequences.
\begin{center}
  \begin{tabular}{cl}
    \verb|\a| & alert \\
    \verb|\b| & backspace \\
    \verb|\e| & ASCII escape \\
    \verb|\f| & formfeed \\
    \verb|\n| & newline \\
    \verb|\r| & carriage return \\
    \verb|\t| & tab \\
    \verb|\v| & vertical tab \\
  \end{tabular}
\end{center}

Any character not in the above table following a backslash will be that
character without the backslash. Thus, a \verb|\"| will be a quote
character.
The quotes surounding the string are not printed.

The arithmetic operators for \emph{string} are:
\begin{center}
  \begin{tabular}{c}
    \verb|-| \emph{string} \\
    \emph{string} \verb|+| \emph{string} \\
    \emph{string} \verb|-| \emph{string} \\
    \emph{real} \verb|*| \emph{string} \\
    \emph{string} \verb|*| \emph{real} \\
  \end{tabular}
\end{center}
The negation operator reverses the order of the string.
The \verb|+| operator will concatenate the two string.
The subtraction operator concatenates the first string and the reverse of the second string.
The \verb|*| operator repeates the string the number of times specified by the real value.
If the real value is negative, then it repeates the reverse of the string the number of times
specified by the absolute value of the real value.
\begin{center}
  \begin{tabular}{cc}
    expression & result \\
    \verb|-"abc"| & \verb|"cba"| \\
    \verb|"abc" + "def"| & \verb|"abcdef"| \\
    \verb|"abc" - "def"| & \verb|"abcfed"| \\
    \verb|3*"abc"| & \verb|"abcabcabc"| \\
    \verb|"abc"*-3| & \verb|"cbacbacba"| \\
  \end{tabular}
\end{center}

The comparison operators for string return are:
\begin{center}
  \begin{tabular}{c}
    \emph{string} \verb|==| \emph{string} \\
    \emph{string} \verb|!=| \emph{string} \\
    \emph{string} \verb|<| \emph{string} \\
    \emph{string} \verb|<=| \emph{string} \\
    \emph{string} \verb|>| \emph{string} \\
    \emph{string} \verb|>=| \emph{string} \\
  \end{tabular}
\end{center}
The string comparison operators work only on the character codes for each character in the strings.
So \verb|"Z" < "a"| is true and \verb|"29" > "8"| is false. 

\subsection{List}
A list is an array of zero of more objects.
Each member of a list can be any type independent of the other members, including another list.
A list is written with curly braces surounding a comma seperated list of objects.
\begin{verbatim}
{ (1.2, 3.4), [1,2,3|4,5,6], "abc" }
\end{verbatim}

All operators operate on lists in a special way.
Unary operators on a list results in a list a list where that unary operator is
applied to evey member of that list.
\begin{verbatim}
-{5, "abc", [1,-2,3]}
\end{verbatim}
results in\footnote{The quotes around the string are not printed.}
\begin{verbatim}
{-5, "cba", [-1, 2, -3]}
\end{verbatim}

Binary operators have two forms.
On form is both operands are lists.
\begin{center}
  \emph{list} \emph{op} \emph{list}
\end{center}
The result of this form is a list with the operator applies
to each member of both operands. If the operand have different
sizes, the members of the smaller list are repeated.
\begin{verbatim}
{1, 2, 3} + { 1, 2, 3, 4, 5}
\end{verbatim}
results in
\begin{verbatim}
{2, 4, 6, 5, 7}
\end{verbatim}

The other form of binary operators is mixing a list operand with a non-list operand.
In this case the result is a list where the singular value is operated on each member
of the list operand.
\begin{verbatim}
2*{1, 2, 3}
\end{verbatim}
results in
\begin{verbatim}
{2, 4, 6}
\end{verbatim}

\section{Indexing}
Individul members of vectors, matricies, string and list can be accessed
by indexes. Only variables and indexed variables can be indexed, not
general expressions.

\subsection{Simple Index}
A simple index selectes one member from the object.
Simple indexes are written by the variable name followed by an index surrounded
by square brackets.
\begin{verbatim}
a[2]
\end{verbatim}
All indexes are zero based so an index of 1 gives the second member of the object.
A simple index of a matrix variable results in a vector of that row.
A member of a matrix at a given row and column can given by two indexs
\begin{verbatim}
m[r][c]
\end{verbatim}
Or you could use matrix indexes.
When elements of a list can be indexed you can use multiple indexes as well.

\subsection{Matrix Index}
A matrix index can access member of a specific row and column.
These indexes are written using a single pair of square brackets with
a row and column index inside, seperated by a comma.
\begin{verbatim}
m[r,c]
\end{verbatim}
If the row index is blank then a vector of the specified column is returned.
If the column index is blank then the row is returned.

If \verb|M| is a matrix:
\begin{center}
  \begin{tabular}{cccl}
    \verb|M[1][2]| & or & \verb|M[1,2]| & returns the value at row 1 and column 2 \\
    \verb|M[1,]| & or & \verb|M[1]| & returns a vector of row 1 \\
    && \verb|M[,2]| & return a vector of column 2 \\
  \end{tabular}
\end{center}

\subsection{Slices}
Slices are use to specify parts of an indexable type.
For example, given a vertor,
\begin{center}
\verb|v = [1, 2, 3, 4, 5, 6]|
\end{center}
then the following expression results in:
\begin{center}
\begin{tabular}{ccc}
  \verb|v[1:3]| & $\Rightarrow$ & \verb|[2, 3, 4]| \\
  \verb|v[1:3:2]| & $\Rightarrow$ & \verb|[2, 4, 6]| \\
\end{tabular}
\end{center}
The first number in a slice is the initial index for the slice.
The second number is the number of members in a slice.
The optional third number is the stride, i.e., the difference between
adjacent indexes.
If the third number is not present, the stride defaults
to one.
Slices can be used for any type and in any way that regular indexes occure.
The commands:
\begin{verbatim}
a = "abcdef"
b = "ABCDEF"
a[1:3:2] = b[0:3:2]
a
\end{verbatim}
will print out:
\begin{verbatim}
aAcCeE
\end{verbatim}
\section{Expressions}
\begin{center}
  \begin{tabular}{cl}
    \emph{number} & real number \\
    \emph{string} & string \\
    \emph{var} & variable \\
    \emph{var} \verb|[| \emph{expr} \verb|]| & simple index \\
    \emph{var} \verb|[| \emph{expr} \verb|,| \emph{expr} \verb|]| & matrix index \\
    \emph{func}\verb|(| \emph{expr-list} \verb|)| & function \\
    \verb|(| \emph{expr} \verb|,| \emph{expr} \verb|)| & complex number \\

    \verb|[| \emph{expr-list} \verb|]| & vector \\
    \verb|[| \emph{row-list} \verb|]| & matrix \\
    \verb|{| \emph{expr-list} \verb|}| & list \\
  \end{tabular}
\end{center}
\begin{center}
  \begin{tabular}{cll}
    \verb|(| \emph{expr} \verb|)| & grouping  \\ \hline
    \emph{expr} \verb|^| \emph{expr} & exponentation & right \\ \hline
    \verb|-| \emph{expr} & negation & left \\ \hline
    \emph{expr} \verb|*| \emph{expr} & multiplication & left \\
    \emph{expr} \verb|/| \emph{expr} & division & left \\
    \emph{expr} \verb|div| \emph{expr} & floored division & left \\
    \emph{expr} \verb|mod| \emph{expr} & modulus & left \\ \hline
    \emph{expr} \verb|+| \emph{expr} & addition & left \\
    \emph{expr} \verb|-| \emph{expr} & subtraction & left \\ \hline
    \emph{expr} \verb|==| \emph{expr} & equal, $=$ & none \\
    \emph{expr} \verb|!=| \emph{expr} & no equal, $\ne$ & none \\
    \emph{expr} \verb|<| \emph{expr} & less than, $<$ & none \\
    \emph{expr} \verb|<=| \emph{expr} & less than or equal, $\le$ & none \\
    \emph{expr} \verb|>| \emph{expr} & greater than, $>$ & none \\
    \emph{expr} \verb|>=| \emph{expr} & greater than or equal, $\ge$ & none \\ \hline
    \verb|not| \emph{expr} & logical \emph{not} & left \\ \hline
    \emph{expr} \verb|and| \emph{expr} & logical \emph{and} & left \\ \hline
    \emph{expr} \verb|or| \emph{expr} & logical \emph{or} & left \\
  \end{tabular}
\end{center}
\section{Commands}
\subsection{Print Command}
The \emph{print} command is just an expression on a line by itself.
\begin{center}
  \emph{expression}
\end{center}
The expression is evaluated and its value printed.
For formated ouput see the standard library.
\subsection{Assignment Command}
The \emph{assignemnt} command evaluates an expresion and
assigns its value to a variable.
\begin{center}
  \emph{lvalue} = \emph{expression}
\end{center}
The \emph{lvalue} can be either a simple variable or an indexed variable.
If the \emph{lvalue} is a simple variable the previous value and
type of the variable are discarded for the new value.
If the \emph{lvalue} is an indexed variable the type (including sizes)
of the expression must match the lvalue.
\subsection{Include Command}
The \emph{include} command is use to the include a file as if
that file replaces the include command.
\begin{center}
  \verb|include| \emph{expression}
\end{center}
The parameter to the include command can be any expression that results in a string.
\subsection{Function Definition}
A function definition command is a multi-lined command used to define functions.
The first line of a function defintion gives the name and parameters of the function.
\begin{center}
  \verb|function| \emph{name} \verb|(| \emph{argument-list} \verb|)|
\end{center}
If the function take no arguments, the empty parenthisis are still required.
The function definition ends with an \emph{end} statement.
\begin{center}
  \verb|end|
\end{center}
In between the first and last line of a function definition are the statements that make up the function.
\section{Variables}
\emph{MPL} is a loosely typed language.
Variables can be assigned any type.
If a variable is assigned a new value, that value need not be same
same type as the previous value.

There are three kinds of variables. Global variables, local variables and function arguments.
Global variables are defined (given a value) outside a function.
Their scope is from the initial assignment till the end of the prorgram.

Function arguments are defined in the first line of a function definition.
Their scope is till the end of the function.
Function arguments are ``pass by reference'', meaning the if you assign a new value to the argument the new value is reflected in the variable passed to the function.
If the value of a parameter is an expression and not a variable, then reassigning the parameter in a function does not change any variable outside the function.
If the following code as executed:
\begin{verbatim}
function test(x)
    x = "hello, world"
end
a = 5
test(a)
a
\end{verbatim}
then
\begin{verbatim}
"hello, world"
\end{verbatim}
will be printed. However, the following code:
\begin{verbatim}
function test(x)
    x = "hello, world"
end
a = 5
test(a+0)
a
\end{verbatim}
will print
\begin{verbatim}
5
\end{verbatim}

Local variable, like function parameters, have scope limited to the function.
If a variable has not been defined as a global variable, assigning a value
to it in a function defines it as a local variable.
If a local variable is latter defined as a global variable, it remains a local variable within the function.

\section{Statements}
The \emph{print} and \emph{assignment} commands also work as statements in a function.
\subsection{If Statement}
\begin{center}
  \begin{tabular}{l}
    \verb|if| \emph{expression} \verb|then| \\
    \quad\quad \emph{statements} \\
    \verb|else if| \emph{expression} \verb|then| \\
    \quad\quad \emph{statements} \\
    \verb|else| \\
    \quad\quad \emph{statements} \\
    \verb|end| \\
  \end{tabular}
\end{center}
The \emph{if} statement only executes statements following it if \emph{expression} evaluates to true.
An \emph{else if} clause will execute the statements following it if a previous \emph{if} or \emph{else if} clause evaluates to false and the its own expression evaluates to true. An \emph{else} clause only executes the statements following it if the previous \emph{if} or \emph{else if} clause evalutates to false.
The \emph{end} clause end any conditional executation. There can be 0 or more \emph{else if} and \emph{else} parts in an \emph{if} statement.
\subsection{While Statement}
\begin{center}
  \begin{tabular}{l}
    \verb|while| \emph{expression} \\
    \quad\quad \emph{statements} \\
    \verb|end| \\
  \end{tabular}
\end{center}
\subsection{Repeat Statement}
\begin{center}
  \begin{tabular}{l}
    \verb|repeat| \\
    \quad\quad \emph{statements} \\
    \verb|until| \emph{expression} \\
  \end{tabular}
\end{center}
\subsection{For Statement}
\begin{center}
  \begin{tabular}{l}
    \verb|for| \emph{variable} \verb|in| \emph{range-specifier} \\
    \quad\quad \emph{statements} \\
    \verb|end| \\
  \end{tabular}
\end{center}
The code
\begin{verbatim}
for x in {"a", "b", "c"}
    x
end
\end{verbatim}
prints
\begin{verbatim}
a
b
c
\end{verbatim}
The code
\begin{verbatim}
for i in 1:3
    i
end
\end{verbatim}
prints
\begin{verbatim}
1
2
3
\end{verbatim}
The code
\begin{verbatim}
for i in 1:5:2
    i
end
\end{verbatim}
prints
\begin{verbatim}
1
3
5
\end{verbatim}
\subsection{Return Statement}
\begin{center}
  \verb|return| \emph{expression}
\end{center}
The \emph{return} statement ends the execution of a function and specifies the return
value.
The return value is optional.
If the executions of a function reaches the end of the function there is an implicite return statement and the function ends without a return value.
Functions that return without a value can only be used in \emph{print} statements/commands.
\section{Standard Library}
\begin{tabular}{lp{8cm}}
  \verb|ceil(x)| & Ceiling function. Rounds $x$ to smallest integer greater than or equal to $x$. \\
\end{tabular}
\section{Things You're Not Suppose to Know.}
This section is for documenting ``undocumneted'' features of \emph{MPL}.
These are features of the language that may change in the future.
\subsection{Real Numbers}
\emph{Real} numbers can be written in hexadecimal form.
A \verb|#| (pound symbol) followed by hexidecimal digits give the bits
of the floating point number. For example, \verb|#3ff0000000000000|
represents the number 1.

The way the current interpreter reads \emph{real} literals,
a \verb|.| (decimal point) by itself represents the number
zero.

\end{document}
